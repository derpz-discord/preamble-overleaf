\documentclass{article}
\usepackage{graphicx} % Required for inserting images
\usepackage{preamble}


\title{Lee Introduction To Manifolds Exercises}
\author{Robert}
\date{February 2023}



\begin{document}



\maketitle

\section{Exercise 2.4}

\begin{definition}
\metricspace{M}{d}

Then, 
$\ball{r}{d}{x} = \set{y \in M : d(y, x) < r}$ is the \textbf{open ball} of radius $r$ around $x$ 
\end{definition}

\begin{definition}
Let $(M, d)$ be a metric space.

Then, 
$\cball{r}{d}{x} = \set{y \in M : d(y, x) \leq r}$ is the \textbf{closed ball} of radius $r$ around $x$ 
\end{definition}

\begin{definition}
    Let $(M, d)$ be a metric space. Then, we call $\mc T_d$ the metric topology generated
    by $d$ (the set of all possible unions of open sets in $M$)
    \footnote{It is important to note that in metric spaces, all open sets are unions of open balls}
\end{definition}

\subsection{Exercise 2.4 (a)}


\begin{theorem}
\label{thm:ex2.4a}
Let $M$ be a metric space with metrics $d$ and $d'$. Then, the metric topologies 
generated by $d$, denoted as $\mc T_d$ and $d'$, denoted as $\mc T_{d'}$
are equivalent, $\mc T_d = \mc T_{d'}$ iff for every $x \in M$ and every $r > 0$
there exists $r_1, r_2 > 0$ such that $\ball{r_1}{d'}{x} \subseteq \ball{r}{d}{x}$ and
$\ball{r_2}{d}{x} \subseteq \ball{r}{d'}{x}$
\end{theorem}
\begin{proof}
\forwarddir Suppose $\mathcal{T}_d = \mathcal{T}_{d'}$. Let $x \in M$ and $r > 0$ be arbitrary.
Since $\mc T_d = \mc T_{d'}$, we have that $\ball{r}{d}{x} \in \mc T_{d'}$. So, $\ball{r}{d}{x}$
is the union of open balls in $\mc T_{d'}$. But this means we must have some $r_1 > 0$ such that $\ball{r_1}{d'}{x} \subseteq \ball{r}{d}{x}$. It's trivial to see that
$\ball{r}{d'}{x} \in \mc T_d$ which again means it is the union of open balls in $\mc T_d$,
thus implying we must have some $r_2 > 0$ such that $\ball{r_2}{d}{x} \subseteq \ball{r}{d'}{x}$. 
 This completes the proof of the forward direction.

% \conversedir Suppose that for every $x \in M$ and every $r > 0$
% there exists $r_1, r_2$ such that $\ball{r_1}{d'}{x} \subseteq \ball{r}{d}{x}$ and
% $\ball{r_2}{d}{x} \subseteq \ball{r}{d'}{x}$. Let $O^{(d)} \in \mc T_d$ be an arbitrary open
% set in $\mc T_d$. Our goal now is to show that $O^{(d)} \in \mc T_{d'}$ which will let us
% conclude that $\mc T_d \subseteq T_{d'}$.
% Let $\Lambda \subseteq \bR \times M$ be a set of radius-point pairs.
% So, $O^{(d)} = \bigcup_{(r, x) \in \Lambda} \ball{r}{d}{x}$

\conversedir  Suppose that for every $x \in M$ and every $r > 0$
there exists $r_1, r_2$ such that $\ball{r_1}{d'}{x} \subseteq \ball{r}{d}{x}$ and
$\ball{r_2}{d}{x} \subseteq \ball{r}{d'}{x}$. Suppose for contradiction that $\mc T_d \neq \mc T_{d'}$. So, either we have an $O^{(d)}$ such that $O^{(d)} \notin \mc T_{d'}$ or we have
$O^{(d')}$ such that $O^{(d')} \notin \mc T_d$. Let's do case 1 first, that $O^{(d)} \notin \mc T_{d'}$. 
% Let $\Lambda \subseteq \bR \times M$ be a set of radius-point pairs.
% So, $O^{(d)} = \bigcup_{(r, x) \in \Lambda} \ball{r}{d}{x}$
Since $O^{(d)}$ is not open in $\mc T_{d'}$, there is an $x_0 \in O^{(d)}$ such that 
\begin{align}
\label{ex2.4a:contrahypo}
    \forall j > 0, \quad \ball{j}{d'}{x_0} \nsubseteq O^{(d)}
\end{align}


Since $O^{(d)}$ is open in $\mc T_d$, for this particular $x_0$, we have $r_0 > 0$ such that $\ball{r_0}{d}{x_0} \subseteq O^{(d)}$. 
By hypothesis, we now have an $r_1 > 0$ such that $\ball{r_1}{d'}{x_0} \subseteq \ball{r_0}{d}{x_0}$.
But now, we have that $\ball{r_1}{d'}{x_0} \subseteq \ball{r_0}{d}{x_0} \subseteq O^{(d)}$, but this contradicts \cref{ex2.4a:contrahypo}. The argument for the other case is similar 
and left out because it is tedious.
\end{proof}

\subsection{Exercise 2.4 (b)}
\begin{claim}
Let $(M, d)$ be a metric space, let $c > 0$ and define $d'(x, y) = c \cdot d(x, y)$. 
Then, $\mc T_d = \mc T_d'$
\end{claim}

\begin{proof}
Let $O \in \metrictop{d}$. Let $\Lambda \subseteq \bR \times M$ be the radius-point pairs
such that $O = \bigcup_{(r, x) \in \Lambda} \ball{r}{d}{x}$.
Let $(r, x) \in \Lambda$ be arbitrary. Expanding the definition of $\ball{r}{d}{x}$, we
get that 
\begin{align*}
    \ball{r}{d}{x} &= \set{y \in M: d(y, x) < r} \\
    &= \set{y \in M:  c \cdot d(y, x) < cr} \\
    &= \set{y \in M:  d'(y, x) < cr} \\
    &= \ball{cr}{d'}{x}
\end{align*}

So, $O = \bigcup_{(r, x) \in \Lambda} \ball{cr}{d'}{x} \in \metrictop{d'}$. 
This shows that $\metrictop{d} \subseteq \metrictop{d'}$. The argument
$\metrictop{d} \supseteq \metrictop{d'}$ is similar and tedious and so skipped.
\end{proof}

Alternative proof by using \Cref{thm:ex2.4a}
\begin{proof}
    Since $d'(x, y) = c \cdot d(x, y)$, we have $\frac{d'(x, y)}{c} = d(x, y)$
    Pick $r_1 = \f r/c.$. Then $\ball{r_1}{d'}{x} \subseteq \ball{r}{d}{x}$
    Now, for $r_2 = cr$, it is immediately obvious that $\ball{r_2}{d}{x} \subseteq \ball{r}{d'}{x}$ \footnote{the balls are actually equal but whatever}
\end{proof}

\subsection{Exercise 2.4 (c)}

This is exercise B.1 but I included it here because it's useful.
\begin{lemma}
\label{lemma:exb.1}
For $x \in \bR^n$ such that $x = (x_1, x_2, \dots, x_n)$, we have
\begin{align*}
\max\set{\abs{x_1}, \abs{x_2}, \dots, \abs{x_n}} \leq \abs x \leq \sqrt n \max\set{\abs{x_1}, \abs{x_2}, \dots, \abs{x_n}}
\end{align*}

\end{lemma}

\begin{proof}
    Suppose $\abs{x_i} = \max\set{\abs{x_1}, \abs{x_2}, \dots, \abs{x_n}}$ for $1 \leq i \leq n$.
    Then, $\abs x = \sqrt{{x_1}^2 + {x_2}^2 + \dots + {x_i}^2 + \dots + {x_n}^2} \geq \sqrt{{x_i}^2} = \abs{x_i}$.
    For the next bit, we have that $\abs x = \sqrt{{x_1}^2 + {x_2}^2 + \dots + {x_i}^2 + \dots + {x_n}^2} \leq \sqrt{n \cdot {x_i}^2} = \sqrt n \cdot \abs{x_i}$
\end{proof}

\begin{claim}
    Let $(\bR^n, d)$ be Euclidean $n$-space with the Euclidean metric $d(x, y) = \sqrt{(x_1 - y_1)^2 + (x_2 - y_2)^2 + \dots + (x_n - y_n)^2 }$
    
    Let $d'(x, y) = \max \set{\abs{x_1 - y_1}, \abs{x_2 - y_2}, \dots, \abs{x_n - y_n} }$ be 
    defined for $\bR^n$. Then, $\metrictop{d} = \metrictop{d'}$
\end{claim}
\textbf{Note: This proof is incomplete since I am confused by my justification}
\begin{proof}
    We will make use of \Cref{thm:ex2.4a}. Let $x \in \bR^n$ and $r > 0$. Then, choose $r_1 = r$. Now,
    let $y \in \ball{r_1}{d'}{x}$. So, from \cref{lemma:exb.1} we have $d'(y, x) < r_1 = r \leq d(y, x)$.
    So, $y \in \ball{r}{d}{x}$ and thus $\ball{r_1}{d'}{x} \subseteq \ball{r}{d}{x}$.
    Choose $r_2 = r\sqrt{n}$. Let $y \in \ball{r_2}{d}{x}$. 
    Now
    \begin{align*}
        &d'(y, x) < r \\
        \implies &\sqrt{n} \cdot d'(y, x) < r\sqrt{n} \\
        \implies &d(x, y) \leq \sqrt{n} \cdot d'(y, x) < r\sqrt{n} \\
        \implies &\f {d(x,y)}/\sqrt{n}. \leq d'(y, x) < r
    \end{align*}
    Again, by \cref{lemma:exb.1}, $d(y, x) < r_2 \leq d'(y, x) < r$ Thus $\ball{r_2}{d}{x} \subseteq \ball{r}{d'}{x}$
\end{proof}


\section{Exercise 2.5}

\begin{claim}
    Let $X$ be a topological space and $Y$ be an open subset of $X$. Then, the 
    collection of all open subsets of $X$ in $Y$ is a topology on $Y$.
\end{claim}

\begin{proof}
    Let $(X, \mc T)$ denote the topological space of $X$.
    Let $\mc O_Y$ denote the collection of all open subsets of $X$ in $Y$.
    To do so, we just need to check the properties of a topology on $Y$, namely, that
    $\varnothing, Y \in \mc O_Y$, if $O_1, O_2 \subseteq Y$ are open, then $O_1 \cap O_2$ is 
    open in $Y$ (aka, $O_1 \cap O_2 \in \mc O_Y$), and the arbitrary union $\bigcup_{\lambda \in \Lambda} O_\lambda$ is also open in $Y$.

    Firstly, since $\emptyset$ and $Y$ are obviously open in $Y$ by definition, $\emptyset, Y \in \mc O_Y$. Now, let $O_1, O_2 \subseteq Y$ be arbitrary open subsets of $Y$. Since $Y$ is open in $X$, any subset of $Y$ must also be open in $X$,
    so this means that $O_1, O_2 \in \mc T$. And since $\mc T$ is a topology, $O_1 \cap O_2 \in \mc T$. Of course, $O_1 \cap O_2 \subseteq Y$, so $O_1 \cap O_2 \in \mc O_Y$. (The inductive
    case is trivial)

    Now, let $\Lambda$ be some indexing set such that for each $\lambda \in \Lambda$,
    $O_{\lambda}$ is open in $Y$. Of course, this means that $\bigcup_{\lambda \in \Lambda} O_{\lambda} \subseteq Y$ and so the arbitrary union is open in Y. And since 
    $\bigcup_{\lambda \in \Lambda} O_{\lambda} \in \mc T$, $\bigcup_{\lambda \in \Lambda} O_{\lambda} \in \mc O_Y$. 

    Since $\mc O_Y$ is demonstrated to have the properties of a topology on $Y$, the proof
    is complete.
\end{proof}

\section{Exercise 2.6}

\begin{claim}
    Let $X$ be a set, and let $\set{\mc T_\alpha}_{\alpha \in A}$ be a collection of 
    topologies on $X$. Then, $\mc T = \medcap_{\alpha \in A} \mc T_\alpha$ is a topology on $X$.
\end{claim}

\begin{proof}
    Obviously, $\emptyset, X \in \mc T$.
    \newpara
    
    Let $O_1, O_2 \in \mc T$.
    So, $\forall \alpha \in A \, O_1, O_2 \in \mc T_\alpha$. But since each $\mc T_\alpha$ 
    is a topology, we know that $O_1 \cap O_2 \in \mc T_\alpha$. So obviously, $O_1 \cap O_2 \in \mc T$.
    \newpara

    Let $\Lambda$ be an indexing set such that $O_\lambda \in \mc T$ for $\lambda \in \Lambda$.
    Now, this means each $O_\lambda \in \mc T_\alpha$.
    But of course, since each $\mc T_\alpha$ is a topology, $\medcup_{\lambda \in \Lambda} O_\lambda \in \mc T_\alpha$. And since $\mc T_\alpha$ was arbitrary, $\medcup_{\lambda \in \Lambda} O_\lambda \in \mc T$.
\end{proof}


\section{Exercise 2.9}

\textbf{Prove proposition 2.8}

\begin{definition}
    \label{def:nbhd}
    Let $(X, \mc T)$ be a topological space. Let $x \in X$. Then $\nbhd{x}$ is a neighborhood
    of $x$ meaning that $\nbhd{x} \in \mc T$ (or that it is an open subset of $X$)
\end{definition}

Let $(X, \mc T)$ be a topological space and let $A \subseteq X$. Let $A^c = X \setminus A$

\begin{proposition}{Textbook Proposition 2.8a}
\label{book:prop:2.8a}
    $x \in \Interior A$ iff there exists $\nbhd{x} \subseteq A$
\end{proposition}
\begin{proof}
    \forwarddir Let $x \in \Interior A$. Then, by definition of $\Interior A$, we have some
    $C \subseteq X$ such that $C \subseteq A$ and $C \in \mc T$ ($C$ is open). Then $C$ will
    serve as $\nbhd{x}$.

    \conversedir trivial
\end{proof}

\begin{proposition}{Textbook Proposition 2.8b}
\label{book:prop:2.8b}
    $x \in \Exterior A$ iff there exists $\nbhd{x} \subseteq A^c$
\end{proposition}
\begin{proof}
    \forwarddir 
    \begin{align*}
        \overline{A}^c &= \paren{\bigcap_{\lambda \in \Lambda} B_\lambda}^c  &&\text{Each $B_\lambda$ is closed} \\
        &= \bigcup_{\lambda \in \Lambda} \paren{B_\lambda}^c && \text{And now each $\paren{B_\lambda}^c$ is open}
    \end{align*}
    The rest is trivial
    
    \conversedir trivial
    
\end{proof}
\begin{proposition}{Textbook Proposition 2.8c}
\label{book:prop:2.8c}
    $x \in \Boundary A$ iff for every $\nbhd{x}$, some $y_1 \in A \cap \nbhd{x}$ and 
    some $y_2 \in A^c \cap \nbhd{x}$
\end{proposition}
\begin{proof}
    \forwarddir. Let $x \in \Boundary A$. Let $\nbhd{x} \in \mc T$ be arbitrary.

    Here, note that $C_\lambda$ are open in $X$ and are subsets of $A$.
    And that $B_\gamma$ are closed in $X$ and contain $A$
    \begin{align*}
        \Boundary A &= \paren{\Interior A \cup \Exterior A}^c \\
        &= \paren{\bigcup_{\lambda \in \Lambda} C_\lambda \cup \overline{A}^c}^c \\ 
        &= \paren{\bigcup_{\lambda \in \Lambda} C_\lambda}^c \cap \overline{A} \\
        &= \bigcap_{\lambda \in \Lambda} \paren{C_\lambda}^c \cap \overline{A} \\
        &= \bigcap_{\lambda \in \Lambda} \paren{C_\lambda}^c \cap \bigcap_{\gamma \in \Gamma} B_\gamma
    \end{align*}
\end{proof}
\begin{proposition}{Textbook Proposition 2.8d}
\label{book:prop:2.8d}
    $x \in \overline{A}$ iff every $\nbhd{x}$ has $y \in A \cap \nbhd{x}$
\end{proposition}
\begin{proof}
    \forwarddir Let $x \in \overline{A}$. Let $\mc F = \set{B_\lambda: \lambda \in \Lambda}$
    be a collection of every closed set in $X$ that contains A, so $\overline{A} = \medcap \mc F$.
    Suppose for contradiction that $\nbhd{x} \in \mc T$ is a neighborhood that contains
    no points of A, or in other words, $\nbhd{x} \cap A = \emptyset$. Since $\nbhd{x}$ is open
    and disjoint from $A$, $A \subseteq X\setminus \nbhd{x}$. So, $X \setminus \nbhd{x}$ is a
    closed set that contains A. Thus, $X \setminus \nbhd{x} \in \mc F$. But since $x \in \overline{A} = \medcap \mc F$, $x \in X \setminus \nbhd{x}$. And by hypothesis, $x \in \nbhd{x}$. This is obviously absurd.

    \conversedir Suppose $x$ is such that every $\nbhd{x}$ has a $y \in A \cap \nbhd{x}$.
    Let $\mc F = \set{B_\lambda: \lambda \in \Lambda}$
    be a collection of every closed set in $X$ that contains A, so $\overline{A} = \medcap \mc F$.
\end{proof}
\begin{proposition}{Textbook Proposition 2.8e}
\label{book:prop:2.8e}
    $\overline{A} = A \cup \Boundary A = \Interior A \cup \Boundary A$
\end{proposition}
\begin{proof}
    \begin{align*}
        \Boundary A = X \setminus \Interior A \cap  
    \end{align*}
\end{proof}
\begin{proposition}{Textbook Proposition 2.8f}
\label{book:prop:2.8f}
    $\Interior A, \Exterior A \in \mc T$ and $\overline{A}^c, \paren{\Boundary A}^c \in \mc T$.
    In other words, the interior and exterior are open sets, while the closure and the boundary and closed sets.
\end{proposition}
\begin{proof}
    $\Interior A$ is the union of open sets which is an open set.

    $\overline{A}$ is the intersection of closed sets, which is a closed set.

    $\Exterior A = X \setminus \overline{A}$ is an open set because $\overline{A}$ is closed.

    $\Boundary A = X \setminus \paren{\Interior A \cup \Exterior A}$ is closed because
    $\Interior A \cup \Exterior A$ is open.
\end{proof}

\begin{proposition}{Textbook Proposition 2.8g}
\label{book:prop:2.8g}
    The following are equivalent:
    \begin{itemize}
        \item $A$ is open in $X$
        \item $A = \Interior A$
        \item $A$ contains none of its boundary points (aka, $\Boundary A \cap A = \emptyset$)
        \item For all $x \in A$, there exists $\nbhd{x} \subseteq A$
    \end{itemize}
\end{proposition}

\begin{proof}
    Suppose $A$ is open in $X$. Then $A \in \mc T$. Obviously, $\Interior A \subseteq A$.
    $A \subseteq \Interior A$ follows from the definition of $\Interior A$, namely, since,
    $A \subseteq A$ and $A$ is open in $X$.

    Suppose $A = \Interior A$.
    Then by definition of $\Boundary A = X \setminus \paren{\Interior A \cup \Exterior A}$,
    $\Boundary A \cap A = \emptyset$ immediately follows.

    Suppose that $\Boundary A \cap A = \emptyset$. Let $x \in A$ be arbitrary.
    This means that $x \notin \Boundary A$, so $x \in \Interior A \cup \Exterior A$. Of course,
    $x \notin \Exterior A$. So, $x \in \Interior A$. So, $\Interior A$ can be a neighborhood of
    $x$, and obviously since $\Interior A \subseteq A$ we are done.

    Suppose that every point of $A$ has a neighborhood contained in $A$. Then, we know
    $\medcup_{x \in A} \nbhd{x} \subseteq A$. Now, let $x \in A$. Then we have some $\nbhd{x}$,
    so $\nbhd{x} \subseteq \medcup_{y \in A} \nbhd{y}$. And so $\medcup_{y \in A} \nbhd{y} = A$.
    Since the union of open sets is open, $A$ is open.
\end{proof}

\begin{proposition}{Textbook Proposition 2.8h}
\label{book:prop:2.8h}
     The following are equivalent:
    \begin{itemize}
        \item $A$ is closed in $X$
        \item $A = \overline{A}$
        \item $A$ contains all of its boundary points (aka, $\Boundary A \subseteq A$)
        \item For all $x \in A^c$, there exists $\nbhd{x} \subseteq A^c$
    \end{itemize}
\end{proposition}

\begin{proof}
    Observe that $A$ is closed iff $A^c$ is open, and apply \cref{book:prop:2.8g}
\end{proof}

\section{Exercise 2.9}

\begin{proposition}{Book Exercise 2.10}
\label{book:ex:2.10}
    Let $(X, \mc T)$ be a topological space and let $A \subseteq X$. Then, $A$ is closed
    iff it contains all of its limit points
\end{proposition}
\begin{proof}
    \forwarddir Suppose $A$ is closed. Let $p$ be a limit point of $A$.
    Suppose for contradiction that $p \notin A$, so $p \in A^c$.
    Since $A^c$ is open, $p$ must have some neighborhood in $A^c$,
    call it $V$. Since $p$ is a limit point, we must have some point $a_0 \in A$ such
    that $a_0 \in V$ and $a_0 \neq p$. 
    But then, $V \subseteq A^c$, so $a_0 \in A$ and $a_0 \in A^c$, which is absurd.

    Suppose $A$ contains all of its limit points. We shall use \cref{book:prop:2.8d} and
    \cref{book:prop:2.8h} to show that $A=\overline{A}$ and thus $A$ is closed.
    
    
\end{proof}


\end{document}
